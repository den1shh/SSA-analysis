\documentclass{beamer}

% Кодировка и язык
\usepackage[T2A]{fontenc}   % Поддержка кириллицы
\usepackage[utf8]{inputenc}  % UTF-8 кодировка
\usepackage[russian]{babel}  % Русский язык

% Тема и цветовая схема
\usetheme{Madrid}
\usecolortheme{dolphin}

% Математика
\usepackage{amsmath, amssymb}

% Информация о презентации
\title{Прогнозирование сегмента временного ряда методом сингулярного анализа спектра}
\author{Гудков Денис}
\institute{ФПМИ МФТИ}
\date{}

\begin{document}

% Титульный слайд
\begin{frame}
  \titlepage
\end{frame}

% Слайд с содержанием
% \begin{frame}{Содержание}
%   \tableofcontents
% \end{frame}

\section{Введение}

\begin{frame}{Цель работы}
    \begin{block}{}
        Рассказать о прогнозировании сегмента временного ряда методом сингулярного анализа спектра. Проанализировать исходную размерность матрицы Ганкеля 
        и ее сниженную размерность.    
    \end{block}
%   \begin{itemize}
%     \item Показать работу Beamer с русским языком
%     \item Поддержка формул
%     \item Красивое оформление слайдов
%   \end{itemize}
\end{frame}

\begin{frame}{Применение сингулярного разложения (SSA)}

\begin{block}{}
SSA позволяет:
\begin{itemize}
    \item Прогнозировать значения временного ряда
    \item Выделять тренды и периодичность
    \item Удалять шум
    \item Разделять детерминистические и стохастические компоненты
\end{itemize}
\end{block}

\end{frame}


\begin{frame}{Постановка задачи прогнозирования с помощью SSA}

Пусть задан временной ряд:
\[
X = \{x_1, x_2, \dots, x_N\}, \quad x_t \in \mathbb{R}.
\]
Необходимо предсказать будущие значения:
\[
x_{N+1}, x_{N+2}, \dots, x_{N+M}.
\]
\end{frame}

\begin{frame}{Построение матрицы Ганкеля}
        Выбираем длину окна \(L\) и формируем матрицу траекторий:
        \[
        \mathbf{X} =
        \begin{pmatrix}
        x_1 & x_2 & \dots & x_K \\
        x_2 & x_3 & \dots & x_{K+1} \\
        \vdots & \vdots & \ddots & \vdots \\
        x_L & x_{L+1} & \dots & x_N
        \end{pmatrix}, \quad K = N - L + 1.
        \]
        Данная матрица называются матрицей Ганкеля. 
    Рекомендации по выбору $L$:
    \begin{itemize}
        \item  Слишком маленькое $L$: плохо выделяются сезонные колебания, шум может доминировать
        \item  Слишком большое $L$: тяжёлые вычисления, важные колебания становятся менее четкими
        \item  Сезонные компоненты: лучше выбирать $L\ge$ периода ряда

    \end{itemize}
\end{frame}

\begin{frame}{Сингулярное разложение}
        Пусть $\mathbf{S} = \mathbf{X} \mathbf{X}^{\mathrm{T}}$, а $\lambda_1, \dots, \lambda_L$ — собственные значения матрицы $\mathbf{S}$, $U_1, \dots, U_L$ — 
        ортонормированный базис из собственных векторов матрицы $\mathbf{S}$, соответствующих этим собственным значениям.
        Пусть  $d = \mathrm{rank} X$ и
        $V_i = \mathbf{X}^{\mathrm{T}} U_i / \sqrt{\lambda_i}, \quad i = 1, \dots, d$.

        В этой записи SVD матрицы $\mathbf{X}$ можно записать как
        \begin{equation}
        \mathbf{X} = \mathbf{X}_1 + \dots + \mathbf{X}_d,
        \end{equation}
        где
        \begin{equation}
        \mathbf{X}_i = \sqrt{\lambda_i}\, U_i V_i^{\mathrm{T}}.
        \end{equation}
        Векторы $\sqrt{\lambda_i} V_i = \mathbf{X}^{\mathrm{T}} U_i$
        называются векторами главных компонент. Векторы \(U_i\) называются левыми сингулярными векторами матрицы \(\mathbf{X}\), а числа \(\sqrt{\lambda_i}\) являются её 
        сингулярными значениями.
\end{frame}

\begin{frame}{Перегруппировка компонент}
        Множество индексов $\{1, \ldots, d\}$ делится на $m$ непересекающихся подмножеств $I_1, \ldots, I_m$.

        Пусть
        \[
        I = \{i_1, \ldots, i_p\}.
        \]
        Тогда результирующая матрица $\mathbf{X}_I$, соответствующая группе $I$, определяется как
        \[
        \mathbf{X}_I = \mathbf{X}_{i_1} + \ldots + \mathbf{X}_{i_p}.
        \]

        Результирующие матрицы вычисляются для групп
        \[
        I = I_1, \ldots, I_m,
        \]
        и сгруппированное SVD-разложение $\mathbf{X}$ можно записать как
        \[
        \mathbf{X} = \mathbf{X}_{I_1} + \ldots + \mathbf{X}_{I_m}.
        \]


\end{frame}

\begin{frame}{Как выбираются компоненты для перегруппировки}
        \begin{enumerate}
            \item \textbf{Анализ сингулярного спектра.} Смотрим на убывание $\sqrt{\lambda_i}$. Резкие скачки говорят о границах между трендом, сезонностью и шумом. 
                    Малые сингулярные значения убираются. 
            \item \textbf{Анализ левых сингулярных векторов.} Гладкие компоненты без колебаний соответствуют тренду, синусоидальные и идущие парами --- сезонности, 
                    хаотичные --- шуму. Пары компонент объединяются в одну группу. 
        \end{enumerate}
        Итоговая группировка состоит из компоненты тренда и сезонных компонент. 

\end{frame}

\begin{frame}{Сравнение исходной и перегруппированной матриц Ганкеля}

\textbf{После SVD и перегруппировки компонент:}
\[
\mathbf{X} = \mathbf{X}_{I_1} + \mathbf{X}_{I_2}
\]

\begin{itemize}
    \item \(\mathbf{X}_{I_1}\) — тренд  
    \item \(\mathbf{X}_{I_2}\) — сезонность (пары гармоник)  
\end{itemize}

Исходная матрица \(\mathbf{X}\) содержит весь ряд, включая шум, а перегруппированная матрицавыделяет только значимые структурные компоненты. 
Снижение размерности упрощает анализ и повышает точность прогноза

\end{frame}


\begin{frame}{Построение линейного рекуррентного соотношения}

После перегруппировки компонент SSA формируем восстановленный ряд:
\[
\mathbf{X}_{\text{recon}} = \sum_{i \in I} \mathbf{X}_i
\]

\textbf{Линейное рекуррентное соотношение} для группы компонент имеет вид:
\[
y_t = a_1 y_{t-1} + a_2 y_{t-2} + \dots + a_{L-1} y_{t-L+1}, 
\quad t = L, \dots, N
\]

Коэффициенты $a_1, \dots, a_{L-1}$ вычисляются через левые сингулярные векторы $U_i$ группы $I$:
\[
A = \frac{1}{1 - \nu^2} \sum_{i \in I} \pi(U_i) \, \underline{U_i}^\top,
\]
где 
\(\underline{U_i} = (u_{i1}, \dots, u_{i,L-1})^\top\), 
\(\pi(U_i) = u_{iL}\), 
\(\nu^2 = \sum_{i \in I} u_{iL}^2\).

% \vspace{0.3cm}
% \textbf{Прогноз будущих значений} строится рекурсивно:
% \[
% \hat{y}_{N+1} = a_1 y_N + \dots + a_{L-1} y_{N-L+2}, \quad
% \hat{y}_{N+2} = a_1 \hat{y}_{N+1} + \dots
% \]

% \vspace{0.3cm}
\footnotesize{Источник: N. Golyandina, A. Zhigljavsky, \textit{Singular Spectrum Analysis for Time Series}, Springer, 2013.}

\end{frame}

\begin{frame}{Итоговый прогноз временного ряда}

После построения линейного рекуррентного соотношения:

\begin{enumerate}
    \item Для каждой группы $I_j$ строится прогноз $ \hat{y}_{N+t}^{(I_j)} $:
    \[
    \hat{y}_{N+t}^{(I_j)} = a_1^{(j)} \, y_{N+t-1}^{(I_j)} + \dots + a_{L-1}^{(j)} \, y_{N+t-L+1}^{(I_j)}, \quad t=1, \dots, h
    \]
    
    \item Итоговый прогноз ряда получается суммированием прогнозов по группам:
    \[
    \hat{x}_{N+t} = \sum_{j=1}^{m} \hat{y}_{N+t}^{(I_j)}, \quad t = 1, \dots, h
    \]
    \end{enumerate}

\end{frame}

\begin{frame}{}
  \begin{figure}
    \centering
    \includegraphics[width=0.95\textwidth]{spectrum.png}
  \end{figure}
\end{frame}

\begin{frame}{}
  \begin{figure}
    \centering
    \includegraphics[width=0.95\textwidth]{pricesgraph.png}
  \end{figure}
\end{frame}

\end{document}
